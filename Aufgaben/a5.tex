\documentclass[a4paper,twoside,12pt]{article}
%\usepackage [reqno] {amsmath}
\usepackage[utf8]{inputenc}
\usepackage[]{algorithm2e}
\usepackage{listingsutf8}
\usepackage{amsfonts,amstext}
\usepackage{amsmath,amssymb}
\usepackage{german}
\usepackage{fullpage}
\usepackage[normalem]{ulem}
\usepackage{todonotes}

\newcommand{\ZETTELNUMMER}[1]{#1}
\newcommand{\ABGABEDATUM}[1]{#1}

\newcounter{AUFGNR}
\setcounter{AUFGNR}{1}
\newcommand{\AUFGABE}[2]{\vspace{0.3cm}\item[Aufgabe~\arabic{AUFGNR}]\stepcounter{AUFGNR} #1\hfill\emph{#2}}


\newcommand{\floor}[1]{\left\lfloor{#1}\right\rfloor}
\newcommand{\ceil}[1]{\left\lceil{#1}\right\rceil}
\newcommand{\eps}{\varepsilon}
\newcommand{\E}{\mathbf{E}}
\renewcommand{\Pr}{\mathbf{Pr}}
\newcommand{\R}{\mathbb{R}}
\newcommand{\N}{\mathbb{N}}
\newcommand{\half}[1]{\frac{#1}{2}}
\newcommand{\enquote}[1]{``#1''}

\renewcommand{\labelenumi}{(\alph{enumi})}
\renewcommand{\labelenumii}{(\roman{enumii})}

\newcommand{\HEADER}[5]{
	\pagestyle{empty}
	\hrule\medskip
	\rule{0ex}{0ex}\\[-1ex]
	\ZETTELNUMMER{#1}. Aufgabenblatt mit Lösung zur Vorlesung

	\smallskip
	\noindent
	\large
	\textbf{#2}\hfill #3 \\[0.5ex]
	\normalsize
	#4

	\medskip\hrule

	\smallskip
	\noindent
	\textbf{Abgabe} \ABGABEDATUM{#5}

	\vskip 0.5cm
}
\begin{document}
\HEADER{3}{Betriebssysteme}{WiSe 2017}{Sebastian Börner, Christian Hofmann, Simon Auch}{7 Dezember 2017, 10 Uhr}
\begin{description}
\AUFGABE{Nachrichten unterschiedlicher Länge}{5 Punkte}

Erläutern Sie die Begriffe: 
    \begin{itemize}
      \item memory mapped I/O,  
      \item DMA (direct memory access) und 
      \item polling. 
    \end{itemize}
Gegeben sei nun ein Prozessor mit einer 200MHz Taktfrequenz. Eine Polling-Operation 
benötige 400 Taktzyklen. Berechnen Sie die prozentuale CPU-Auslastung durch das Polling für die folgenden Geräte. Bei 2 und 3 ist dabei die Abfragerate so zu wählen, dass keine Daten verloren gehen. 
    \begin{enumerate}[label=\arabic*.]
      \item Maus mit einer Abfragerate von 30/sec, 
      \item Diskettenlaufwerk: Datentransfer in 16-bit-Wörtern mit einer Datenrate von 50KB/sec, 
      \item Plattengerät, das die Daten in 32-bit-Wörtern mit einer Rate von 2MB/sec transportiert. 
    \end{enumerate}
(Hinweis: Wie viele Taktzyklen gibt es pro Sekunde? Wie viele Taktzyklen werden für eine 
Abtastrate von 30/sec benötigt? Was für einer Abtastrate entsprechen 50KB/sec bei 16-bit-
Wörtern ...) 

Für das Plattengerät soll jetzt DMA eingesetzt werden. Wir nehmen an, dass für das 
Initialisieren des DMA 4000 Takte, für eine Interrupt-Behandlung 2000 Takte benötigt werden und dass per DMA 4KB transportiert werden. Das Plattengerät sei ununterbrochen aktiv. 
Zugriffskonflikte am Bus zwischen Prozessor und DMA-Steuereinheit werden ignoriert. 

Wie hoch ist nun die prozentuale Belastung des Prozessors? (Hinweise: Wie viele DMA-
Transfers pro Sekunde sind bei 2MB/sec und 4KB Transfergröße notwendig?) 

\textbf{Lösung:}
    \begin{description}
      \item[memory mapped I/O]
        Teil des Adressraums wird für Speicher und I/O verwendet; Belegt also Teil des verfügbaren Adressraums; I/O Geräte reagieren auf Zugriffen des Bereichs

        Beispiel: Das setzen eines Bits einer bestimmten Adresse sorgt dafür, dass in dieser beim nächsten Lesen ein Zeichen der Benutzereingabe steht.
      \item[DMA]
        Erlaubt Zugriff auf den Speicher ohne I/O -Operationen via CPU nutzen zu müssen.
      \item[polling] Das Abfragen eines Geräts (o.Ä.), ob entsprechendes bereit ist.
        Der Nachteil dieses Vorgehens besteht darin, dass relativ viel Rechenzeit dafür verwendet wird, diese Abfragen zu tätigen. Als gegenteiliges Vorgehen lässt man sich z.B. durch einen Callback über die Bereitschaft informieren.
    \end{description}

Gegeben sei nun ein Prozessor mit einer 200MHz Taktfrequenz. Eine Polling-Operation 
benötige 400 Taktzyklen. Berechnen Sie die prozentuale CPU-Auslastung durch das Polling für die folgenden Geräte. Bei 2 und 3 ist dabei die Abfragerate so zu wählen, dass keine Daten verloren gehen. 
    \begin{enumerate}[label=\arabic*.]
      \item Maus mit einer Abfragerate von 30/sec, 

        30 Abfragen pro Sekunde mit je 400 Taktzyklen ergibt 12kHz. 12kHz/200000kHz = 0,000060 = 0,006\% Auslastung.
      \item Diskettenlaufwerk: Datentransfer in 16-bit-Wörtern mit einer Datenrate von 50KB/sec, 

        50kB (Kilobyte, nicht Kelvin-Byte wie in der Aufgabenstellung) entsprechen 3125 16-Bit-Wörtern. Wir erhalten wieder 400*3125 = 1,25MHz; 1,25MHz/200MHz = 0,00625 = 0,625\%.
      \item Plattengerät, das die Daten in 32-bit-Wörtern mit einer Rate von 2MB/sec transportiert. 

        2MB entsprechen 62500 32-Bit-Wörtern. 62500*400 = 25MHz entspricht 12,5\%.
    \end{enumerate}

    Bei 2MB/s sind 500 Transfers von je 4kB nötig.
    Wir erhalten (4000+2000)*500=3MHz.

    Analog zu den vorherigen Berechnungen erhalten wir 3MHz/200MHz = .015 = 1,5\%.

\AUFGABE{User-/Kernel-Interface}{15 Punkte}

In dieser Aufgabe soll Ihr Betriebssystem um eine einfache Koordination einerseits und 
andererseits um eine effizientere Ressourcen-Nutzung erweitert werden. 

Definieren Sie hierzu eine ordentliche Schnittstelle zwischen Anwendungen und Betriebs-
system und führen Sie blockierende Systemrufe ein, damit wartende Threads nicht unnötig die CPU blockieren. 
\begin{enumerate}[label=\arabic*.]
  \item Legen Sie eine Aufrufkonvention für Systemrufe mittels SWI fest. 
  \item Definieren Sie Systemrufe für die Aufgaben „Zeichen ausgeben“, „Zeichen einlesen“, 
„Thread beenden“, „Thread erzeugen“ und „Thread für bestimmte Zeitspanne ver-
zögern“. 
  \item Implementieren Sie die Systemrufe auf Seite des Kernels. Dabei sollen „Zeichen 
einlesen“ und „Thread verzögern“ blockierend arbeiten. Das heißt, der Thread wird so 
lange nicht mehr durch den Scheduler erfasst, bis ein Zeichen da ist bzw. die 
vorgegebene Zeitspanne vorbei ist. Überlegen Sie sich, ob ein aufwachender oder 
neuer Thread dem gerade laufenden Thread vorgezogen werden sollte oder nicht. 
  \item Implementieren Sie eine „Bibliothek“, die für Anwendungen eine leichter zu benutzende 
Schnittstelle zur Verfügung stellt, als direkt SWIs aufzurufen. Idealerweise sind der 
Code für Anwendungen/Anwendungsbibliotheken und der Code für das Betriebs-
system/Betriebssystembibliotheken zum Schluss vollständig unabhängig voneinander. 
\end{enumerate}
Durch die SWI-Schnittstelle wird zum einen eine logische Trennung von Anwendungen und 
Betriebssystem erzeugt. Zum anderen wird – da nur einen Kern vorhanden ist – automatisch 
ein gegenseitiger Ausschluss sämtlicher Kernel-Funktionen realisiert (sofern Sie in 
privilegierten Modus [Modi] die Interrupts nicht demaskiert, siehe Zusatzaufgabe). 
Die Demonstration ist ähnlich wie in der letzten Aufgabe zu implementieren, diesmal jedoch mit 
eigenem Thread: 
\begin{enumerate}
  \item[5.] Schreiben Sie eine Anwendung, die stets auf Zeichen wartet. Wann immer ein Zeichen 
empfangen wird, erzeugt diese Anwendung einen neuen (Anwendungs-)Thread und 
gibt diesem das empfangende Zeichen als Parameter mit. 
  \item[6.] Der neu erzeugte Thread soll das Zeichen wiederholt (aber nicht endlos) mit kleinen 
Pausen ausgeben und sich anschließend beenden. Handelt es sich bei dem Zeichen 
um einen Großbuchstaben wird zur Erzeugung der Pause aktiv gewartet (wie bisher), 
bei anderen Zeichen wird die Pause durch den neu eingeführten Systemruf erzeugt. 
\end{enumerate}
Bei den daraus resultierenden Ausgaben ist insbesondere das zeitliche Verhalten spannend. 
Die folgenden Beispiele gehen davon aus, dass die Rechenzeit beim aktiven Warten genauso 
lang ist, wie die Dauer der Verzögerung via Systemruf. (Und beides ist ein Vielfaches der 
Zeitscheibenlänge.) 
\begin{description}
  \item[2x aktives Warten:] ABABABABABABABABABABAB 
  \item[1x aktiv + 1x passiv:] AbAbAbAbAbAbAbAbAbAbAb 
  \item[2x passiv:] ababababababababababab 
  \item[1x aktiv + 2x passiv:] AbcAbcAbcAbcAbcAbcAbcAbcAbcAbcAbc 
  \item[2x aktiv + 1x passiv:] ABcAcBcAcBcAcBcAcBcAcBABABABAB
\end{description}

\textbf{Lösung:}
Anleitung zum Erstellen eines mit qemu benutzbaren Kernels:
\begin{enumerate}
  \item Wechseln in den Ordner \texttt{GrandiOS}
  \item Zunächst wird mit \texttt{make setup} die Toolchain von Rust vorbereitet. Unter Umständen sind Benutzereingaben erwartet, diese können einfach mit Enter bestätigt werden.
  \item Alternativ kann mittels \texttt{make update} die Toolchain und alle abhängigkeiten aktualisiert werden.
  \item Ausführen von \texttt{make run} zum Erstellen eines Kernels und starten von Qemu mit diesem.
\end{enumerate}
Das Vorbereiten der Toolchain mittels \texttt{make setup} ist nur einmal notwendig.

Für den unwahrscheinlichen Fall, dass aus irgendeinem Grund das Bauen des Kernels fehlschlägt, ist unter {\texttt{GrandiOS/kernels/kernel\_05}} noch ein von uns gebauter Kernel vorhanden.

Die Aufgabenstellung kann getestet werden indem in der Shell \texttt{u5} ausgeführt wird.

Eine kleine Übersicht, wo die für diese Aufgabe relevanten Codesegmente zu finden sind:
\begin{itemize}
	\item \texttt{GrandiOS/swi/src/lib.rs} Gibt die Aufrufkonvention für Systemaufrufe mittels SWI vor. Grob zusammengefasst muss in r0 ein Zeiger auf ein Struct für die Ausgabe des Aufrufs und in r1 ein Zeiger auf ein Struct für die Eingaben des Aufrufs stehen. Die jeweiligen Structs werden mit den Aufrufen \texttt{build\_swi!} erstellt.
	\item \texttt{GrandiOS/aids/src/lib.rs} Stellt die Abstraktionsschicht über die Systemaufrufe dar.
	\item \texttt{GrandiOS/src/utils/exceptions/software\_interrupt.rs} Beinhaltet die Abarbeitung der Systemaufrufe. Von hier werden entsprechend die Funktionen anderer Teile des Kernels aufgerufen.
	\item \texttt{GrandiOS/shell/src/commands/u5.rs} Beinhaltet die Funktion der Shell die entsprechenden Threads zu starten, als auch die Funktion die der neue Thread ausführen soll.
\end{itemize}

\end{description}
\end{document}
