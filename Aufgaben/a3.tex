\documentclass[a4paper,twoside,12pt]{article}
%\usepackage [reqno] {amsmath}
\usepackage[utf8]{inputenc}
\usepackage[]{algorithm2e}
\usepackage{listingsutf8}
\usepackage{amsfonts,amstext}
\usepackage{amsmath,amssymb}
\usepackage{german}
\usepackage{fullpage}
\usepackage[normalem]{ulem}
\usepackage{todonotes}

\newcommand{\ZETTELNUMMER}[1]{#1}
\newcommand{\ABGABEDATUM}[1]{#1}

\newcounter{AUFGNR}
\setcounter{AUFGNR}{1}
\newcommand{\AUFGABE}[2]{\vspace{0.3cm}\item[Aufgabe~\arabic{AUFGNR}]\stepcounter{AUFGNR} #1\hfill\emph{#2}}


\newcommand{\floor}[1]{\left\lfloor{#1}\right\rfloor}
\newcommand{\ceil}[1]{\left\lceil{#1}\right\rceil}
\newcommand{\eps}{\varepsilon}
\newcommand{\E}{\mathbf{E}}
\renewcommand{\Pr}{\mathbf{Pr}}
\newcommand{\R}{\mathbb{R}}
\newcommand{\N}{\mathbb{N}}
\newcommand{\half}[1]{\frac{#1}{2}}
\newcommand{\enquote}[1]{``#1''}

\renewcommand{\labelenumi}{(\alph{enumi})}
\renewcommand{\labelenumii}{(\roman{enumii})}

\newcommand{\HEADER}[5]{
	\pagestyle{empty}
	\hrule\medskip
	\rule{0ex}{0ex}\\[-1ex]
	\ZETTELNUMMER{#1}. Aufgabenblatt mit Lösung zur Vorlesung

	\smallskip
	\noindent
	\large
	\textbf{#2}\hfill #3 \\[0.5ex]
	\normalsize
	#4

	\medskip\hrule

	\smallskip
	\noindent
	\textbf{Abgabe} \ABGABEDATUM{#5}

	\vskip 0.5cm
}
\begin{document}
\HEADER{3}{Betriebssysteme}{WiSe 2017}{Sebastian Börner, Christian Hofmann, Simon Auch}{7 Dezember 2017, 10 Uhr}
\begin{description}
\AUFGABE{Threads}{6 Punkte}

Grenzen Sie die folgenden Begriffe auf Grund der Vorlesung gegeneinander ab: 
    \begin{itemize}
      \item gegenseitiger Ausschluss 
      \item Signalisierung 
      \item Synchronisation 
      \item Koordination 
      \item Kommunikation 
      \item Kooperation 
    \end{itemize}
Gehen Sie bei Ihrer Diskussion insbesondere auf die Teilnehmer und ihre Beziehung 
zueinander ein. 

\textbf{Lösung:}\\

    \begin{description}
      \item[gegenseitiger Ausschluss]
        Gegenseitiger Ausschluss dient dazu sicherzustellen, dass nicht zwei oder mehr Prozesse/Threads zeitgleich auf eine bestimmte Ressource zugreifen.
      \item[Signalisierung ]
        Signalisierung wird benutzt um zum Beispiel bei gegenseitigen Ausschluss anderen Prozessen/Threads, die auf eine vom aktuellen Prozess/Thread beanspruchte Ressource warten, zu signalisieren, dass diese wieder zu Verfügung steht.
      \item[Synchronisation ]
        Synchronisation ist ein Mechanismus, der sicherstellt, dass mehrere Threads sich an einer definierten Stelle in ihrem Ablauf befinden.
      \item[Koordination ]
        Bei Koordination wird entschieden, wer wann worauf zugreifen kann. Bei Threads mit geteilter Ressource können dabei u.A. mehrere Threads (Leser) für die Ressource so koodiniert werden, dass diese zeitgleich drauf zugreifen können, sofern kein Schreiber für die Ressource grade aktiv ist (also anders als beim gegenseitigen Ausschluss wo je nur einer zugelassen wird).
      \item[Kommunikation ] Kummunikation zwischen Threads (in Bezug auf Betriebssysteme) findet häufig über geteilten Speicher statt. Damit beim geteilten Zugriff auf die Daten keine Fehler entstehen wird für gewöhnlich ein Verfahren für gegenseitigen Ausschluss verwendet, wie das Sperren des Zugriffs auf den gemeinsamen Speicher via eines Mutex, eines Semaphors oder eines Monitors. Ggf muss den anderen Threads signalisiert werden, wenn Daten geschrieben wurden.
      \item[Kooperation ]
        In Bezug auf das Scheduling, ist Kooperation in der Art zu finden, dass Threads sich untereinander Zeit für den Kernel zuteilen, bzw. an andere Threads abgeben, wenn sie eine gewisse (in der Regel selbst definierte) Zeit den Prozessor beansprucht haben.\\
        In Bezug auf das Lösen von Aufgaben von Problemen, können die Threads kooperativ zusammenarbeiten indem mehrere Threads unterschiedliche Teilaufgaben eines Problems angehen und anschließend mittels Kommunikations- und Synchronisationsmechanismen, wie denen von den vorigen Punkte, die Ergebnisse zusammenführen.
        
    \end{description}

\AUFGABE{Interrupts - System Timer und serielle Schnittstelle}{15 Punkte}

Mit dem Abfangen von Ausnahmen ist unser „Betriebssystem“ noch kein wirkliches Betriebs-
system, sondern eher ein Bare-Metal-Programm mit ein wenig Programmier-Komfort. Um 
diesen Umstand auszuräumen, unterstützen wir in dieser Aufgabe auch Hardware-Interrupts. 
Hardware-Interrupts bieten später die einzige Möglichkeit für das Betriebssystem die Kontrolle 
über den Prozessor wiedererlangen zu können, wenn mal ein Prozess in einer Endlosschleife 
hängen sollte. Außerdem geben uns Hardware-Interrupts die Möglichkeit, zeitnah auf Hardware 
zu reagieren, auch wenn der Prozessor gerade eigentlich mit etwas anderem beschäftigt ist. 

Speziell in dieser Aufgabe stellen wir das Lesen (und optional das Schreiben) von der seriellen 
Schnittstelle (bei uns DBGU) von Polling auf Interrupt-getriebene Behandlung um. Interrupt-
getriebenes Lesen verringert die Chance, dass wir ein Zeichen verpassen, weil wir es nicht 
rechtzeitig abgeholt haben. (Interruptgetriebenes Schreiben würde zudem ein Fortsetzen der 
Programmausführung ermöglichen, ohne dass auf die Fertigstellung der Übertragung des 
vorherigen Zeichens gewartet werden muss.) 

Damit sichergestellt ist, dass uns ein Hardware-Interrupt ereilt (auch wenn der Benutzer keine 
Taste drückt), setzen wir außerdem proaktiv den System Timer (ST) ein, um so periodisch 
Interrupts zu erzeugen und unserem Betriebssystem eine garantierte Eingriffsmöglichkeit zu 
geben. 
  

\pagebreak
\textbf{Lösung:}
Anleitung zum erstellen eines mit qemu benutzbaren Kernels:
\begin{enumerate}
  \item Wechseln in den Ordner \texttt{GrandiOS}
  \item Falls die Rust-Umgebung bereits installiert wurde diese mittels \texttt{rustup update} aktualisieren.
  \item Zunächst wird mit dem Script \texttt{setup\_env.sh} die Toolchain von Rust vorbereitet. Unter Umständen sind Benutzereingaben erwartet, diese können einfach mit Enter bestätigt werden.
  \item Ausführen von \texttt{make run} zum erstellen eines Kernels und starten von Qemu mit diesem.
\end{enumerate}
Das vorbereiten der Toolchain mittels \texttt{setup\_env.sh} ist nur einmal notwendig.

Für den unwahrscheinlichen Fall, dass aus irgend einem Grund das bauen des Kernels fehlschlägt, ist unter {\texttt{GrandiOS/kernels/kernel\_03}} noch ein von uns gebauter Kernel vorhanden.

Eine kleine Übersicht wo die für diese Aufgabe relevanten Codesegmente zu finden sind:
\begin{itemize}
	\item \texttt{GrandiOS/src/driver/system\_timer.rs}
	\item \texttt{GrandiOS/src/utils/exceptions/irq.rs}
	\item \texttt{Grandios/src/lib.rs::init}
	\item \texttt{GrandiOS/src/driver/interrupts.rs}
        \item Die Funktionalität ist in dem Programm u3 zu betrachten, welches von der Shell aus gestartet werden kann. Die Ausgabe der Ausrufezeichen erfolgt aus Bequemlichkeit auch außerhalb dessen. Der entsprechende Code hierfür liegt in \texttt{GrandiOS/shell/src/commands/u3.rs}.
          
          Hierbei ist zu bemerken, dass das read!-Makro nicht wie im zweiten Zettel aus dem Code der DebugUnit verwendet wird, sondern dass unsere STD-lib (\texttt{GrandiOS/aids/src/lib.rs}) einen Softwareinterrupt mittels eines für den Kernel und das Userland geteilten Interfaces (\texttt{GrandiOS/swi/src/lib.rs}) auslöst, der dann das von dem Interrupt der DebugUnit gelieferte Zeichen erhält und wieder aufgeweckt wird.
\end{itemize}

\end{description}
\end{document}
