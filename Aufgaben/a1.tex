\documentclass[a4paper,twoside,12pt]{article}
%\usepackage [reqno] {amsmath}
\usepackage[utf8]{inputenc}
\usepackage[]{algorithm2e}
\usepackage{listingsutf8}
\usepackage{amsfonts,amstext}
\usepackage{amsmath,amssymb}
\usepackage{german}
\usepackage{fullpage}
\usepackage[normalem]{ulem}
\usepackage{todonotes}

\newcommand{\ZETTELNUMMER}[1]{#1}
\newcommand{\ABGABEDATUM}[1]{#1}

\newcounter{AUFGNR}
\setcounter{AUFGNR}{1}
\newcommand{\AUFGABE}[2]{\vspace{0.3cm}\item[Aufgabe~\arabic{AUFGNR}]\stepcounter{AUFGNR} #1\hfill\emph{#2}}


\newcommand{\floor}[1]{\left\lfloor{#1}\right\rfloor}
\newcommand{\ceil}[1]{\left\lceil{#1}\right\rceil}
\newcommand{\eps}{\varepsilon}
\newcommand{\E}{\mathbf{E}}
\renewcommand{\Pr}{\mathbf{Pr}}
\newcommand{\R}{\mathbb{R}}
\newcommand{\N}{\mathbb{N}}
\newcommand{\half}[1]{\frac{#1}{2}}
\newcommand{\enquote}[1]{``#1''}

\renewcommand{\labelenumi}{(\alph{enumi})}
\renewcommand{\labelenumii}{(\roman{enumii})}

\newcommand{\HEADER}[5]{
	\pagestyle{empty}
	\hrule\medskip
	\rule{0ex}{0ex}\\[-1ex]
	\ZETTELNUMMER{#1}. Aufgabenblatt mit Lösung zur Vorlesung

	\smallskip
	\noindent
	\large
	\textbf{#2}\hfill #3 \\[0.5ex]
	\normalsize
	#4

	\medskip\hrule

	\smallskip
	\noindent
	\textbf{Abgabe} \ABGABEDATUM{#5}

	\vskip 0.5cm
}
\begin{document}
\todo[inline]{title}

\section*{Aufgabe 1}
(Paper: Towards Real \(\mu\)-Kernels (1996))
\todo[inline]{2.Paper}

\todo[inline]{Aufgabe ist bzgl makro nicht monolithisch \(\rightarrow\) einordnen usw.}
\(\mu\)-Kernel im Bezug zu monolithischen

\begin{itemize}
  \item nur IPC, MMU, Scheduler Teil des Kernels
  \item Interrupts werden ausserhalb des Kernels behandelt
  \item[+] nur Kernel kann sicherheitskritische Operationen eines Prozessors nutzen, software in user mode nicht
  \item[+] Treiberausfälle etc. sind \emph{nur} Softwarefehler
  \item[+] modularer/flexibel/leicht erweiterbar da nicht Kernel für neue Geräte angepasst/erweitert werden muss
  \item[+] Kernel besser wartbar, da kleiner
  \item[+] Treiber etc. nur Zugriff auf zugewiesenen Speicherbereichen
  \item[-] ineffizienter / mehr overhead bei IPCs, Addressraum wechsel etc.
  \item[-] Je nach Hardwarezugriff können nun (leicht austauschbare) Treiber (im user mode) das System korrumpieren
\end{itemize}

\end{document}
