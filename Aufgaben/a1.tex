\documentclass[a4paper,twoside,12pt]{article}
%\usepackage [reqno] {amsmath}
\usepackage[utf8]{inputenc}
\usepackage[]{algorithm2e}
\usepackage{listingsutf8}
\usepackage{amsfonts,amstext}
\usepackage{amsmath,amssymb}
\usepackage{german}
\usepackage{fullpage}
\usepackage[normalem]{ulem}
\usepackage{todonotes}

\newcommand{\ZETTELNUMMER}[1]{#1}
\newcommand{\ABGABEDATUM}[1]{#1}

\newcounter{AUFGNR}
\setcounter{AUFGNR}{1}
\newcommand{\AUFGABE}[2]{\vspace{0.3cm}\item[Aufgabe~\arabic{AUFGNR}]\stepcounter{AUFGNR} #1\hfill\emph{#2}}


\newcommand{\floor}[1]{\left\lfloor{#1}\right\rfloor}
\newcommand{\ceil}[1]{\left\lceil{#1}\right\rceil}
\newcommand{\eps}{\varepsilon}
\newcommand{\E}{\mathbf{E}}
\renewcommand{\Pr}{\mathbf{Pr}}
\newcommand{\R}{\mathbb{R}}
\newcommand{\N}{\mathbb{N}}
\newcommand{\half}[1]{\frac{#1}{2}}
\newcommand{\enquote}[1]{``#1''}

\renewcommand{\labelenumi}{(\alph{enumi})}
\renewcommand{\labelenumii}{(\roman{enumii})}

\newcommand{\HEADER}[5]{
	\pagestyle{empty}
	\hrule\medskip
	\rule{0ex}{0ex}\\[-1ex]
	\ZETTELNUMMER{#1}. Aufgabenblatt mit Lösung zur Vorlesung

	\smallskip
	\noindent
	\large
	\textbf{#2}\hfill #3 \\[0.5ex]
	\normalsize
	#4

	\medskip\hrule

	\smallskip
	\noindent
	\textbf{Abgabe} \ABGABEDATUM{#5}

	\vskip 0.5cm
}
\begin{document}
\HEADER{1}{Betriebssysteme}{WiSe 2017}{Basse, Christian, Simon}{9 November 2017, 10 Uhr}
\begin{description}
\AUFGABE{Architekturen}{5 Punkte}

Vorhandene Betriebssystementwürfe können grob in zwei Klassen eingeteilt werden:
\begin{itemize}
	\item die Makrokernarchitektur,
	\item die Mikrokernarchitektur.
\end{itemize}
Disktuieren Sie beide Architekturansätze unter Zuhilfenahme mindestens folgender quellen:
\footnotesize{\begin{itemize}
	\item J. Liedtke, Toward real $\mu$-kernels, Communications of the ACM, 39(9):70--77, September 1996, \\http://citeseerx.ist.psu.edu/viewdoc/download?doi=10.1.1.40.4950\&rep=rep1\&type=pdf
	\item C. Maeda, B.N. Bershad, Networking performance for microkernels, Proceedings of Third Workshop on Workstation Operating Systems, 13:154 – 159, April 1992 \\http://citeseerx.ist.psu.edu/viewdoc/download?doi=10.1.1.56.9733\&rep=rep1\&type=pdf 
\end{itemize}}

\textbf{Lösung:}\\
\todo[inline]{Aufgabe ist bzgl makro nicht monolithisch \(\rightarrow\) einordnen usw.}
\(\mu\)-Kernel im Bezug zu monolithischen Kernel

\begin{itemize}
  \item nur IPC, MMU, Scheduler Teil des Kernels
  \item Interrupts werden ausserhalb des Kernels behandelt
  \item Protokolle (z.B. UDP) werden ausserhalb des Kernels implementiert
  \item Implementationen von Protokellen, etc. von monolithischen Kerneln können unter umständen nicht ohne gravierende geschwindigkeits Verluste \emph{direkt} übernommen werden
  \item[+] nur Kernel kann sicherheitskritische Operationen eines Prozessors nutzen, software in user mode nicht
  \item[+] Treiberausfälle etc. sind \emph{nur} Softwarefehler
  \item[+] modularer/flexibel/leicht erweiterbar da nicht Kernel für neue Geräte angepasst/erweitert werden muss
  \item[+] Kernel besser wartbar, da kleiner
  \item[+] Treiber etc. nur Zugriff auf zugewiesenen Speicherbereichen
  \item[-] ineffizienter / mehr overhead bei IPCs, Addressraum wechsel etc.
  \item[-] Höhere Latenz zum bearbeiten von Daten, da (mindestens) ein Kontextwechsel nach dem Auslösen eines Interrupts stattfinden muss
  \item[-] Je nach Hardwarezugriff können nun (leicht austauschbare) Treiber (im user mode) das System korrumpieren
\end{itemize}


\AUFGABE{Steuerung von Geräten}{15 Punkte}

Implementieren Sie eine \texttt{print}-Funktion zum ausgeben von Daten über die serielle Schnittstelle (Debug-Unit, DBGU).

\textbf{Lösung:}

Anleitung zum erstellen eines mit qemu benutzbaren kernels:
\begin{enumerate}
	\item Wechseln in den Ordner \texttt{GrandiOS}
	\item Zunächst wird mit dem Script \texttt{setup\_env.sh} die Toolchain von Rust vorbereitet. Unter umständen sind benutzereingaben erwartet, diese können einfach mit Enter bestätigt werden.
	\item Durch starten von \texttt{kernel\_build.sh} kann nun der Kernel \texttt{kernel} erzeugt werden.
	\item Starten von qemu mit dem Kernel: \texttt{qemu-bsprak -piotelnet -kernel kernel}
	\item Starten einer telnet Verbindung zu qemu: \texttt{telnet localhost 44444}
\end{enumerate}
Das vorbereiten der Toolchain mittels \texttt{setup\_env.sh} ist nur einmal notwendig.

Für den unwahrscheinlichen Fall, dass aus irgend einem Grund das bauen des Kernels fehlschlägt, ist unter \texttt{GrandiOS/kernels/kernel\_01} noch ein von uns gebauter Kernel vorhanden.

\end{description}
\end{document}
